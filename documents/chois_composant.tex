\documentclass[5pt]{article}
\usepackage{array}
\usepackage{listings}
\usepackage{graphicx}
\usepackage{color}
\usepackage{hyperref}
\lstset{
  basicstyle=\fontfamily{lmvtt}\selectfont\small,
  columns=fullflexible,
}
\title{Choix composant}
\author{TUELEAU Tom}
\begin{document}
\maketitle

\section{Introduction}

Le but de ce document est de mettre par ecris les possibiliter en ce qui concerne les composant pour la premier partie du stage. L'objectif de cette premier partie est le suivant (Extrait ordre de mission).\\ 
\\
Nous avons besoin de mettre en place une première installation avec un microcontroleur et
des capteurs dans la ruche afin de faire un "proof of concept" simple pour pouvoir afficher les
données enregistrée sur le site internet où les vidéos des abeilles seront diffusées.\\
Les données à enregistrer seront :\\
— Température\\
— Hygrométrie\\
— Capteur de vibration fixé sur une gaufre de la ruche\\
Elles devront être remontées en MQTT sur un serveur déjà mis en place.
Le tout doit être opérationnel (fonctionnel et dans la ruche) avant le 15 Mai 2022).
\\
\\
Danc ce document je commencerait par présenter les differente possibilité aux niveaux des micro-controleur. Ensuite j'évoquerait les model disponible des capteur d'hygrometrie et de température. Enfin je reviendrais sur le cas du capteur de vibration.

\newpage{}
\section{Choix des composant}

\subsection{Microcontroleur}
Le micro controleur doit pouvoire :
	- Récolter les données des different capteurs.
	- Envoyer les données via MQTT
	- Mise en place rapide et facile

De ces trois critéres j'ai trouver deux solution possible. 
Tout d'abords un arduino muni d'un shield eternet avec poe pourait nous permetre dans un premier temps d'avoire un systeme alimenter et connecté à internet. 
En second temps j'ai pensé à un Esp32 qui permetrait de faire transiter les données en Wifi. 
Etant familiariser avec les deux solution je n'ai pas de préférence.

\subsection{Capteur de température et d'hygrométrie }
Pour le capteur 
J'ai déjà eux l'occasion d'utiliser plusieurs model du capteur de température.


\subsection{Capteur de vibration}



%\end{lstlisting}
%\begin{center}
%    \begin{tabular}{|l|l|}
%   \end{tabular}
%\end{center}
\end{document}
