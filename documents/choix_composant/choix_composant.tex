\documentclass[5pt]{article}
\usepackage{array}
\usepackage{listings}
\usepackage{graphicx}
\usepackage{color}
\usepackage{hyperref}
\lstset{
  basicstyle=\fontfamily{lmvtt}\selectfont\small,
  columns=fullflexible,
}
\title{Choix composant}
\author{TUELEAU Tom}
\begin{document}
\maketitle

\section{Introduction}

Le but de ce document est de mettre par ecris les possibiliter en ce qui concerne les composant pour la premier partie du stage. Ceux ci ne sont pas defintife et pouront changer tout aux long de la mission. L'objectif de cette premier partie est le suivant (Extrait ordre de mission).\\ 
\\
Nous avons besoin de mettre en place une première installation avec un microcontroleur et
des capteurs dans la ruche afin de faire un "proof of concept" simple pour pouvoir afficher les
données enregistrée sur le site internet où les vidéos des abeilles seront diffusées.\\
Les données à enregistrer seront :\\
— Température\\
— Hygrométrie\\
— Capteur de vibration fixé sur une gaufre de la ruche\\
Elles devront être remontées en MQTT sur un serveur déjà mis en place.
Le tout doit être opérationnel (fonctionnel et dans la ruche) avant le 15 Mai 2022).
\\
\\
Je dois donc concevoire un prototype rapidement afin de répondre aux besoins ennoncer ci-dessus.
Danc ce document vous présenterait les choix fait lors de la premier semaine. Nous commenceront pas voire les differentes possibilité aux niveaux des micro-controleur. Ensuite j'évoquerait les modelles des capteur d'hygrometrie et de température. Enfin je reviendrais sur le cas du capteur de vibration.

\newpage{}
\section{Choix des composant}

\subsection{Microcontroleur}
Le micro controleur doit pouvoire :
	- Récolter les données des different capteurs.
	- Envoyer les données via MQTT
	- Mise en place rapide et facile

De ces trois critéres j'ai trouver deux solution possible. 
Tout d'abords un arduino muni d'un shield ethernet pourrait nous permetre dans un premier temps d'avoire un systeme connecté.
En second temps j'ai pensé à un Esp32 qui permetrait de faire transiter les données en Wifi et qui est plus petits. 
Etant familiariser avec les deux solution je n'ai pas de préférence. J'ai cependant commencer à travailler sur l'arduino.

\subsection{Capteur de température et d'hygrométrie }
Pour répondre à ce besoins j'ai opter pour un Si7021. J'ai fait ce choix car le capteur était directement a disposition et que je l'avais déja programmer. Ces caracteristique sont les suivante :

\\
Température :\\
\begin{center}
    \begin{tabular}{|l|l|}
	\hline
	    Plage de valeur & Résolution \\
	\hline
	    -40C - 125C & 0,4C \\
	\hline
    \end{tabular}
\end{center}


Huniditer:\\
\begin{center}
    \begin{tabular}{|l|l|}
	\hline
	    Plage de valeur & Résolution \\
	\hline
	    0\% - 80\% & 0,3\% \\
	\hline
    \end{tabular}
\end{center}

\subsection{Capteur de vibration}
Le capteur de vibration était la partie que je connaisait pas du tout sur le projet. Je me suis donc rapporcher des Monsieur Druon qui ma fournie deux solution différente. Un capteur piezo-electrique et un microphone. Lors de cette premier semaine 
piezo can amplificationm instrumentation
https://fr.wikipedia.org/wiki/Amplificateur\_de\_mesure

%\end{lstlisting}
%\begin{center}
%    \begin{tabular}{|l|l|}
%   \end{tabular}
%\end{center}
\end{document}
