\documentclass[11pt,french,a4paper]{article}
\usepackage[utf8]{inputenc}
\usepackage[french]{babel}
\usepackage[T1]{fontenc}
\usepackage{fancyhdr}
\usepackage{fancybox}
\usepackage{lastpage}
\usepackage{graphicx}
\usepackage[left=2cm,right=2cm,top=2cm,bottom=2.5cm]{geometry}
\geometry{a4paper}
\setlength{\parindent}{0pt}
\usepackage{listings}
\usepackage{color}
\usepackage[table]{xcolor}
\usepackage{array}
\usepackage{listings}
\usepackage{hyperref}
\usepackage{caption}
\usepackage{lastpage}
\pagestyle{fancy}

\lstset{
  basicstyle=\fontfamily{lmvtt}\selectfont\small,
  columns=fullflexible,
}

\fancyhead[L]{\includegraphics{}}
\fancyhead[C]{Rapport semaine deux et trois du stage }
\fancyhead[R]{zbeub}
\fancyfoot[L]{\small Tom TUELEAU\normalsize}
\fancyfoot[C]{zbeub}
\fancyfoot[R]{zbeub}

\title{Rapport de la semaine du 11 avril 2022}
\author{TUELEAU Tom}
\begin{document}
\maketitle
\section{Introduction}
Ce document a pour objectif de faire l'état d'avancement du stage. Celui-ci résumera donc le travail effectué lors des semaines du 11 avril 2022.
Je vous présente dans un premier temps les recherche que j'ai pu effectuer (Piezo-Electrique, AOP, Frequence des abeilles). Dans une deuxième partie,
je parlerait des simulation effectué sous spice lors de ces 2 semaines. Enfin, une dernière partie introduira les outil que j'ai mis en place pour 
m'organiser et ce conclura par la planification des taches pour la semaines prochaine. 

\section{Recherche}

\newpage
\listoffigures
\end{document}




\begin{center}	
\includegraphics[scale=0.075]{.jpg}
\captionof{figure}{test}
\label{image2}
\end{center}
