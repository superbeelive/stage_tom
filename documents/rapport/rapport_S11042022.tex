\documentclass[5pt]{article}
\usepackage{array}
\usepackage{listings}
\usepackage{graphicx}
\usepackage{color}
\usepackage{hyperref}
\lstset{
  basicstyle=\fontfamily{lmvtt}\selectfont\small,
  columns=fullflexible,
}
\title{Rapport de la semaine du 11 avril 2022}
\author{TUELEAU Tom}
\begin{document}
\maketitle
\section{Introduction}
Ce document à pour objectif de faire l'états d'avancement du stage. Cellui-ci résumera donc le travail effectue la semaine du 11 avril 2022.
Je vous présente dans un premier temps mon instalation et comment j'ai aménager mon espace de travail. Dans une deuxiéme partie
nous verrons le travail que j'ai effectuer sur les capteurs et le microcontroleurs lors des 5 jours. Enfin, une dernier partie introduira le travail 
que je prévois d'effectuer lors des semaines à venir. 

\section{Instalation}
Lors de cette semaine j'ai pus m'installer aux niveaux du rucher. J'y ai apporte du matériel récupérer a l'iut (un osciloscope, shield ethernte pour
arduino uno) et du matériel personnel (arduino, capteurs, cable ...). Nous avons aussi a disposition un switch relier à la fibre. J'ai été, et je serais,
de nouveau amener à me déplacer à béziers pour deux raissons. La premier, afin de récurpérer d materiel et la seconde car le laboratoire à béziers est beaucoup
plus confortable pour manipuler des systeme electroniques.
\section{Choix des composant}

\subsection{Mise en context}

Le but de ce document est de mettre par ecris les possibiliter en ce qui concerne les composant pour la premier partie du stage. Ceux ci ne sont pas defintife et pouront changer tout aux long de la mission. L'objectif de cette premier partie est le suivant (Extrait ordre de mission).\\ 
\\
Nous avons besoin de mettre en place une première installation avec un microcontroleur et
des capteurs dans la ruche afin de faire un "proof of concept" simple pour pouvoir afficher les
données enregistrée sur le site internet où les vidéos des abeilles seront diffusées.\\
Les données à enregistrer seront :\\
— Température\\
— Hygrométrie\\
— Capteur de vibration fixé sur une gaufre de la ruche\\
Elles devront être remontées en MQTT sur un serveur déjà mis en place.
Le tout doit être opérationnel (fonctionnel et dans la ruche) avant le 15 Mai 2022).
\\
\\
Je dois donc concevoire un prototype rapidement afin de répondre aux besoins ennoncer ci-dessus.
Danc ce document vous présenterait les choix fait lors de la premier semaine. Nous commenceront pas voire les differentes possibilité aux niveaux des micro-controleur. Ensuite j'évoquerait les modelles des capteur d'hygrometrie et de température. Enfin je reviendrais sur le cas du capteur de vibration.

\newpage{}
\subsection{Choix des composant}

\subsubsection{Microcontroleur}
Le micro controleur doit pouvoire :
	- Récolter les données des different capteurs.
	- Envoyer les données via MQTT
	- Mise en place rapide et facile

De ces trois critéres j'ai trouver deux solution possible. 
Tout d'abords un arduino muni d'un shield ethernet pourrait nous permetre dans un premier temps d'avoire un systeme connecté.
Dans un second temps j'ai pensé à un Esp32 qui permetrait de faire transiter les données en Wifi et qui est plus petits. 
Etant familiariser avec les deux solution je n'ai pas de préférence. J'ai cependant commencer à travailler sur l'arduino.

\subsubsection{Capteur de température et d'hygrométrie }
Pour répondre à ce besoins j'ai opter pour un Si7021. J'ai fait ce choix car le capteur était directement a disposition et que je l'avais déja programmer. Ces caracteristique sont les suivante :\\
Température :\\
\begin{center}
    \begin{tabular}{|l|l|}
	\hline
	    Plage de valeur & Résolution \\
	\hline
	    -40C - 125C & 0,4C \\
	\hline
    \end{tabular}
\end{center}


Huniditer:\\
\begin{center}
    \begin{tabular}{|l|l|}
	\hline
	    Plage de valeur & Résolution \\
	\hline
	    0\% - 80\% & 0,3\% \\
	\hline
    \end{tabular}
\end{center}

\subsubsection{Capteur de vibration}
Le capteur de vibration est la partie que je connait le moin du projet. Je me suis donc rapporcher de Monsieur Druon qui ma fournie deux solutions . Une premier à l'aide d'un capteur piezo-electrique et une seconde avec un  microphone. Ces deux capteur étant les seul solution que j'avais a ma disposition, j'ai décider de les conserver et de les tester afin de voire si elles pouvaient convenires au projet.\\
Les deux références sont :\\
- p37e pour le piezo-electrique\\
- INMP441 pour le microphone\\

\section{Capteur Piezo-Electrique}


\end{document}
