
\section*{Annexe}
\subsection*{Code}
\subsubsection*{Récolte et envoi des données}
\begin{scriptsize}
\begin{lstlisting}

//Libraries
// Communication serie on utilise :
// minicom -b 9600 -o -D /dev/ttyUC0


#include <SPI.h>
#include <PubSubClient.h>
#include <Ethernet.h>
#include <EthernetUdp.h>
#include <Adafruit_Si7021.h>
#define PORT_MQTT 1883
#define PORT_UDP 8888


void udpSendChar(char * data);
void gatherVibration(char * fdata, int nbEchantillon, int nbVirgule, double periodeEchantillonageMs);
void mqttSendHumTmp(int nbVirgule, Adafruit_Si7021 sensor);
void callback(char* topic, byte* payload, unsigned int length);
int beginEthSen();
int piezo = 0;
Adafruit_Si7021 sensor = Adafruit_Si7021();
String request ;
unsigned long refreshCounter  = 0;
byte mac [6] = {0x54, 0x34, 0x41, 0x30, 0x30, 0x31};
IPAddress ipMqtt(194, 199, 227, 239) ;
IPAddress ipUdp(10, 24, 4, 127) ;
EthernetUDP Udp;
EthernetClient ethclient; 
PubSubClient client(ipMqtt,PORT_MQTT,callback,ethclient); 

void setup() 
{
	Serial.begin(9600);
	if(!beginEthSen()){
		Serial.println("Erreur");
	}
	client.connect("arduino");
}
void loop() 
{
	char data[320];
	mqttSendHumTmp(2,sensor);
	gatherVibration(data,80,2,2);
	delay(1000);
}


// Lance tout les fonction. Ehternet, UDP, MQTT et le capteur. 
int beginEthSen()
{
	int error = 0;
	if (!sensor.begin()) 
	{
		Serial.println("Did not find Si7021 sensor!");
		error=1;
	}

	Serial.println(F("Initialize System"));
	Ethernet.begin(mac);

	while (!Ethernet.begin(mac)) 
	{
		Serial.println(F("failed. Retrying in 1 secondsi."));
		error=1;
	}

	pinMode(2, OUTPUT);
	pinMode(0, INPUT);
	Serial.print(F("IP Address: "));
	Serial.println(Ethernet.localIP());
	Udp.begin(PORT_UDP);
	if(error=1)
		return -1;

	return 0;
}


// Cette fonction recolte et envoie l'humidité et la température en MQTT. nbVirgule permet de spécifier le nombre de valeurs apres la virgule.
void mqttSendHumTmp(int nbVirgule,Adafruit_Si7021 sensor)
{

	double hum,temp;
	char Ctemp[2+nbVirgule];
	char Chum[2+nbVirgule];
	hum = sensor.readHumidity();
	temp = sensor.readTemperature();
	dtostrf(temp,2,nbVirgule,Ctemp);
	dtostrf(hum,2,nbVirgule,Chum);
  
  Serial.println("HUMIDITER ET TEMPERATURE");
	Serial.println(Chum);
	Serial.println(Ctemp);
	if(!client.publish("cadre3/temperature",Ctemp))
	{
		Serial.println("error");
	}
	if(!client.publish("cadre3/humidite",Chum))
	{
		Serial.println("error");
	}

}

void gatherVibration(char * fdata, int nbEchantillon, int nbVirgule, double periodeEchantillonageMs)
{
	// Afin d'avoir le nombre de caractere je fais le nombre de chiffres apres la virgule plus deux
	// qui represente mon premier digit et ma virgule.
	int nbChar = nbVirgule+2;
	char tmp[nbEchantillon][nbChar];
	char data2[nbEchantillon*nbChar];
	double centsValeurs[nbEchantillon];
	int i,y;
	for(i=0;i<nbEchantillon;i++){
		centsValeurs[i]=analogRead(piezo);
		centsValeurs[i]=(centsValeurs[i]*5)/1023;
		delay(periodeEchantillonageMs);
	}
	delay(2000);
	for(i=0;i<nbEchantillon;i++){
		dtostrf(centsValeurs[i],nbChar,nbVirgule,tmp[i]);
		for(y=0;y<nbChar;y++){
			data2[(nbChar*i)+y]=tmp[i][y];
		}	
	}
	Serial.println(data2);
	Udp.beginPacket(ipUdp,PORT_UDP);
	Udp.write(data2);
	Udp.endPacket();
}

void callback(char* topic, byte* payload, unsigned int length){

}
\end{lstlisting}
\end{scriptsize} 
\subsubsection*{Transformmer de Fourrier}
\begin{scriptsize}
\begin{lstlisting}

#include <stdio.h>
#include <complex.h>
#include <fftw3.h>
#include <stdlib.h>
#include <unistd.h>
#include <string.h>
#include <sys/types.h>
#include <sys/socket.h>
#include <arpa/inet.h>
#include <netinet/in.h>
//#include <arduinoUDP.h>

#define PORT	 8888
#define MAXLINE 2048

void traitementData(double * valeurs, char * charData, int taille, int nbChar);


int main (){

	printf("Début du programme. \n");
	int sockfd;
	char buffer[MAXLINE];
	struct sockaddr_in servaddr, cliaddr;
	double data[20];
	if ( (sockfd = socket(AF_INET, SOCK_DGRAM, 0)) < 0 ) {
		perror("Erreur lors de la creation de la socket");
		exit(EXIT_FAILURE);
	}

	memset(&servaddr, 0, sizeof(servaddr));
	memset(&cliaddr, 0, sizeof(cliaddr));

	servaddr.sin_family = AF_INET; 
	servaddr.sin_addr.s_addr = INADDR_ANY;
	servaddr.sin_port = htons(PORT);

	if ( bind(sockfd, (const struct sockaddr *)&servaddr,sizeof(servaddr)) < 0 )
	{
		perror("bind failed");
		exit(EXIT_FAILURE);
	}

	int len, n;

	len = sizeof(cliaddr); //len is value/result
	for(;;){
		n = recvfrom(sockfd, (char *)buffer, MAXLINE,MSG_WAITALL, ( struct sockaddr *) &cliaddr,&len);
		buffer[n] = '\0';
		printf("Client : %s\n", buffer);
		traitementData(data,buffer,20,4);
	}
	return 0;
}

void traitementData(double * valeurs, char * charData, int taille, int nbChar){
	int i,y;
	char tmp[4];
	printf("\n");
	for(i=0;i<taille;i++){
		for(y=0;y<nbChar;y++)
			tmp[y]=charData[(i*nbChar)+y];
		valeurs[i]=atof(tmp);
		printf("%f ",valeurs[i]);
	}
	printf("\n ");
}
\end{lstlisting}
\end{scriptsize} 

