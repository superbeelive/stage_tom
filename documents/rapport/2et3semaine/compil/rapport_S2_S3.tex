\documentclass[11pt,french,a4paper]{article}
\usepackage[utf8]{inputenc}
\usepackage[french]{babel}
\usepackage[T1]{fontenc}
\usepackage{fancyhdr}
\usepackage{fancybox}
\usepackage{lastpage}
\usepackage{graphicx}
\usepackage[left=2cm,right=2cm,top=2cm,bottom=2.5cm]{geometry}
\geometry{a4paper}
\setlength{\parindent}{0pt}
\usepackage{listings}
\usepackage{color}
\usepackage[table]{xcolor}
\usepackage{array}
\usepackage{listings}
\usepackage{hyperref}
\usepackage{caption}
\usepackage{lastpage}
\pagestyle{fancy}

\fancyhead[L]{\includegraphics[width=1cm]{../../../logo/SBLlogo.png}}
\fancyhead[C]{Rapport d'activités semaines du 11 et 18 avril 2022 }
\fancyhead[R]{}
\fancyfoot[L]{\small Tom TUELEAU\normalsize}
\fancyfoot[C]{}
\fancyfoot[R]{\thepage/\pageref{LastPage}}


\lstset{
  basicstyle=\fontfamily{lmvtt}\selectfont\small,
  columns=fullflexible,
}

\title{
 \centering
         \includegraphics[width=4cm]{../../../logo/IUTlogo.png}  \hspace{7cm}
         \includegraphics[width=4cm]{../../../logo/UMlogo.png}  \hspace{7cm}
    
	\LARGE{Rapport d'activités semaines du 11 et 18 avril 2022 }
	\author{TUELEAU Tom}
}
\author{
	\date{}
}
\begin{document}
\maketitle
	 \includegraphics[width=4cm]{../../../logo/LIRMMlogo.png}  \hspace{7cm}
         \includegraphics[width=4cm]{../../../logo/IBMMlogo.jpg}  \hspace{7cm}
\newpage
\tableofcontents
\newpage
\section{Introduction}
Ce document a pour objectif de faire un état d'avancement du stage. Celui-ci résumera donc le travail effectué lors des semaines du 11 et 18 avril 2022.
\\Je vais vous présenter dans un premier temps les recherches que j'ai pu effectuer sur le piezo-electrique les AOP ou encore la fréquence qu'émettent les abeilles. 
\\Dans une deuxième partie, je parlerai de la découverte du logiciel spice ainsi que les simulations que j'ai pu effectuée lors de ces deux semaines.
\\Une troisième partie reviendra sur les résultats obtenus pour le capteur de vibration. 
\\Enfin, une dernière partie introduira les outils que j'ai mis en place pour m'organiser tout au long du projet. 

\newpage
\section{Recherche}
Comme dit en introduction, j'ai été amené à me documenter afin d'accomplir plusieurs objectifs. Le premier objectif était d'étudier les solutions à ma disposition afin d'amplifier mon signal. Mes recherches se sont donc portées sur les amplificateurs opérationnels. Mon second objectif était de faire des recherches sur la fréquence émise par les abeilles afin de commander des capteurs de vibrations avec une plage de fréquence adéquate. 
\subsection{AOP}
Ma principale source d'information pour cette partie a été M. Druon m'ayant fourni plusieurs documentations sur les montages à amplificateur opérationnel. 
Mes recherches ont principalement consisté à tester des montages et à vérifier par le calcul si les valeurs récupérées étaient les bonnes.
Je suis donc passé directement à de l'expérimentation en testant comme vu lors du précédent rapport différents montages basiques.
Nous verrons plus en détail lors de la partie "Capteur de vibration" la mise en pratique de ces montages.

\subsection{Fréquence et piézo-électrique}
Afin de pouvoir mesurer avec plus de précision les vibrations émises par les abeilles, j'ai pris conseil auprès de M. Rousset qui m'a fourni différents articles sur le sujet.
Après une lecture de ces documents, je suis arrivé à une estimation de la plage de fréquence de communication des abeilles dans la ruche à [200Hz-1000Hz]. Je me base sur cet échange avec M. Rousset pour déduire cette plage de valeurs : \\ 
\\
\\
\fbox{
	\begin{minipage}{17cm} 
Les vibrations dans les matériaux qui sont probablement détectées par plusieurs organes/structures dans le corps de l’abeille. Par exemple l’organe subgénual (subgenual organ) localisé dans la patte est sensible aux vibrations entre 200 et 1000Hz.
	\end{minipage}

}
\\ \\ \\Etant donné le temps restant avant l'échéance, j'ai préféré me baser sur ce résultat pour la commande des piézo-électriques.

\newpage
\section{Simulation}
Afin de pouvoir simuler les montages qu'il me sera nécessaire d'effectuer, il m'a été conseillé d'étudier et de me former sur le logiciel spice.

\subsection{ngspice}
Étant sur une distribution linux, j'ai installé le logiciel "ngspice". Celui-ci est Open Source et est installable très facilement sur n'importe quelle distribution basée sur Unix. L'objectif principal pour moi est de pouvoir simuler mes montages afin d'étudier leur comportement. 

\subsection{Apprentissage du logiciel}
Lors de cette semaine d'apprentissage de spice j'ai pu apprendre les choses suivantes :
\\
\\
- Simuler un circuit de base avec résistor, condensateur, bobine, générateur de tension, générateur de courant\\
- Créer un signal continu, sinusoïdale et impulsionel \\
- Afficher les résultats de ceci à l'aide de graphiques \\

\subsection{Etat actuel}
Malgré le temps investi sur celui-ci, je ne maîtrise pas encore parfaitement le logiciel et suis encore en train de le prendre en main. Je compte cependant tenter, dans la semaine à venir, de simuler le montage que vous verrez dans la prochaine partie. 
\newpage
\section{Capteurs de vibration}
Le capteur piezo-electrique ne produisant pas une tension assez élevée pour que je l’utilise tel quel, j’ai donc dû trouver un moyen de l’amplifier. Je vais vous présenter les montages que j’ai pu étudier et les résultats obtenus.

\subsection{Montage}
\subsubsection{Rappel}

Comme vu dans le rapport de la première semaine, j’ai commencé par faire des montages basiques (Suiveur, Amplificateur non-inverseur) composés d’amplificateurs opérationnels (AOP). Ceux-ci m’avaient posé des problèmes dus aux modèles d’AOP que j’utilise (LM741) qui nécessitent symétrique.

\subsubsection{Amplificateur de mesure}

Lors de cette semaine je suis donc passé au montage amplificateur de mesure. Ce montage est très souvent utilisé afin d’amplifier le signal de sortie d’un capteur ayant une tension trop faible.
Il existe plusieurs montages amplificateur de mesures. Dans mon cas, j’ai commencé par un montage nécessitant 3 AOP. Nous pouvons voir ce montage figure \ref{3AOP}

\begin{center}	
\includegraphics[scale=1]{../img/instrumentation3aop_bb.png}
\captionof{figure}{Montage 3 AOP sous Fritzing}
\label{3AOP}
\end{center}


Il existe cependant des montages plus simples notamment un composé de deux amplificateurs opérationnels que l'on peut voir figure \ref{2AOP}.
\\

\begin{center}	
\includegraphics[scale=0.80]{../img/instrumentation2aop_bb.png}
\captionof{figure}{Montage 2 AOP sous Fritzing}
\label{2AOP}
\end{center}

Aprés avoir réussi à effectuer ce montage, j’ai pu remarquer que la sensibilité avait diminué. Ceci a eu pour effet de d’empêcher la visualisation de l’impulsion créée lors de petits frottements. Après plusieurs tentatives de modification pour pallier à ce changement, j’ai décidé de changer de montage.

\subsubsection{Montage actuel}
Le piezo-electrique utilisé jusqu’à maintenant était fourni avec un module. Celui-ci ne changeait rien au signal et pour cause, les composants sur lui n’étaient pas totalement connectés. Cependant, après avoir effectué des modifications sur celui-ci le signal m'a paru beaucoup plus net.
J’ai par la suite effecuté le montage visible figure \ref{MTGFINAL}
\\
\begin{center}	
\includegraphics[scale=0.85]{../img/mtgfinal.png}
\captionof{figure}{Dernier montage }
\label{MTGFINAL}
\end{center}
Il est composé d’un suiveur et d’un amplificateur non-inverseur avec un coefficient multiplicateur de 130. C'est à dire que la tension de sortie sera 130 fois la tension d'entrée. Enfin, un condensateur se trouvant en sortie de circuit nous permet d’éliminer la composante continue ajoutée par les AOP à notre signal.


\subsubsection{Résultat}
Avec le montage vue précédemment nous obtenons plusieurs résultats probants.
Tout d'abord, nous pouvons voir figure \ref{VWOMVM} le signal en sortie du montage quand aucun évènement ne se produit.
Ensuite figure \ref{F}, nous pouvons voir la réaction après un léger frottement effectué avec un câble visible figure \ref{M}. 

\begin{center}	
\includegraphics[scale=0.5]{../img/plat.jpg}
\captionof{figure}{Tension sans mouvement}
\label{VWOMVM}
\end{center}

\begin{center}	
\includegraphics[scale=0.5]{../img/frotment.jpg}
\captionof{figure}{Tension lors d'un Frottement}
\label{F}
\end{center}

\begin{center}	
\includegraphics[scale=0.1]{../img/froty.jpg}
\captionof{figure}{Frottement}
\label{M}
\end{center}

\subsubsection{Conclusion}
Nous sommes donc passés d’une tension de l’ordre dont l’intervalle était [50-90mV] à un intervalle de [0,6-1,5V]. Ces valeurs seront bien plus exploitables et permettent le traitement via l’Arduino.
Cependant, le montage montré ci-dessus ne sera pas le final. Il me faut encore produire un montage plus propre et traiter la partie numérisation du signal.

\newpage
\section{Gestion}
Afin de pouvoir m'organiser et avoir un suivi de mon activité tout au long du stage, j'ai mis en place un Gantt sous la forme d'une page Notion.

\subsection{Organisation de l'espace }
Chaque page que j'ai créé sur notion possède une fonction précise. Je vais donc vous présenter brièvement chacune d'entre elles.
\subsubsection{Calendrier}

Comme son nom l'indique, la page calendrier répertorie chaque tâche ayant été effectuée sous la forme d'un calendrier. Celle-ci me permet de m'organiser sur le mois à venir et surtout de garder un oeil sur l'échéances. 
\begin{center}	
\includegraphics[scale=0.35]{../img/notioncalender.png}
\captionof{figure}{Calendrier}
\label{Calendrier}
\end{center}

\subsubsection{Liste des tâches}
La page liste des tâches regroupe toutes les tâches effectuées et à effectuer. J'attribue à ces tâches des étiquettes ce qui me permet de les classer.   
\begin{center}	
\includegraphics[scale=0.35]{../img/notionlistesdestaches.png}
\captionof{figure}{Liste des tâches}
\label{Liste des taches}
\end{center}

\subsubsection{Description des tâches}
Cet onglet contient une description plus détaillée des tâches. J'y stocke aussi les compte-rendu de réunion, prises de note, images relatives à mes parties du projet.
\begin{center}	
\includegraphics[scale=0.35]{../img/notiondescriptiondestaches.png}
\captionof{figure}{Description des taches}
\label{Description des taches}
\end{center}

\subsubsection{Gantt}
La dernier fiche accessible est la fiche "Gantt" qui me permet surtout de m'organiser sur ma semaine. 
\begin{center}	
\includegraphics[scale=0.35]{../img/notiongantt.png}
\captionof{figure}{Gantt}
\label{Gantt}
\end{center}
\newpage 
\section{Conclusion}

Les tâches qu'il me reste à effectuer pour les prochaines semaines sont les suivantes :
\\
\\- Mettre en place dans la ruche les capteurs déjà opérationels
\\- Rendre opérationel le capteur de vibrations en finalisant sur une plaquette de prototypage les montages d'amplification.
\\- Organiser l'affichage des données avec la personne chargée du site Web. 
\\- Reprendre l'analyse du logiciel "Jaaba" de  Kristin Branson. 
\\- Regarder et mettre en fonctionnement le microphone fourni.
\newpage
\listoffigures
\end{document}
